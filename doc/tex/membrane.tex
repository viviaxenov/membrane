\documentclass[12pt,a4paper,fleqn]{article}
\usepackage[utf8]{inputenc}
\usepackage{amssymb, amsmath, multicol}
\usepackage[russian]{babel}
\usepackage{graphicx}
\usepackage[shortcuts,cyremdash]{extdash}
%\usepackage{wrapfig}
%\usepackage{floatflt}
%\usepackage{lipsum}

\oddsidemargin=-15.4mm
\textwidth=190mm
\headheight=-32.4mm
\textheight=277mm
\tolerance=100
\parindent=0pt
\parskip=8pt
\pagestyle{empty}
\renewcommand{\tg}{\mathop{\mathrm{tg}}\nolimits}
\renewcommand{\ctg}{\mathop{\mathrm{ctg}}\nolimits}
\renewcommand{\arctan}{\mathop{\mathrm{arctg}}\nolimits}
\newcommand{\divisible}{\mathop{\raisebox{-2pt}{\vdots}}}
\RequirePackage{caption2}
\renewcommand\captionlabeldelim{}
\newcommand*{\hm}[1]{#1\nobreak\discretionary{}%
{\hbox{$\mathsurround=0pt #1$}}{}}
\title{Задача про мембрану}
\author{Аксенов Виталий}
\date{$9.7.2018$~---~$\infty$}

\begin{document}

\maketitle
\thispagestyle{empty}



\section{Уравнение движения элемента}
Строим сетку на поверхности мембраны. Запишем второй закон Ньютона для элемента сетки:
\numberwithin{equation}{section}
\begin{equation}
	m\overline{a} = \overline{F}_{el} + \overline{F}_{ext}
\end{equation}
$F_{ext}$~---~внешняя сила, действует не везде, а в месте удара.
Распишем упругую силу $F_{el}$.
Представим поверхность как $w(x,y)$. Базис на поверхности задается векторами $r_x=(1, 0, \frac{\partial w}{\partial x})^\mathbf{T}$,  $r_y=(0, 1, \frac{\partial w}{\partial y})^\mathbf{T}$.
В проекции на эти векторы
\begin{align}
	m\frac{\partial v_x}{\partial t} &= h\delta(\sigma_{xx}^{right} - \sigma_{xx}^{left}) + h\delta(\sigma_{xy}^{right} - \sigma_{xy}^{left}) \\
	m\frac{\partial v_y}{\partial t} &= h\delta(\sigma_{yy}^{top} - \sigma_{yy}^{bottom}) + h\delta(\sigma_{xy}^{top} - \sigma_{xy}^{bottom})
\end{align}

Напряжения выражаются через деформации:
\begin{align}
	\varepsilon_{xx} &= \frac{1}{E}\sigma_{xx} -\frac{\mu}{E}\sigma_{yy} \\
	\varepsilon_{yy} &= \frac{1}{E}\sigma_{yy} -\frac{\mu}{E}\sigma_{xx} \\
	\varepsilon_{xy} &= \frac{1}{G}\sigma_{xy}, \quad G = \frac{E}{2(1 + \mu)}
\end{align}
где $E$~---~модуль Юнга, $\mu$~---~коэффициент Пуассона.
\begin{align}
	\sigma_{xx} &= \frac{E}{1 - \mu^2}\varepsilon_{xx} + \frac{\mu E}{1 - \mu^2}\varepsilon_{yy} \\
	\sigma_{yy} &= \frac{E}{1 - \mu^2}\varepsilon_{yy} + \frac{\mu E}{1 - \mu^2}\varepsilon_{xx} \\
	\sigma_{xy} &= G\varepsilon_{xy}
\end{align}

Пусть $u^*, v^*$~---~координаты в базисе на поверхности мембраны (вблизи какой-то точки $(x_0, y_0)$). Если в тензоре деформаций откинуть квадратичные члены, то:
\begin{align}
	\varepsilon_{xx} &= \frac{\partial u^*}{\partial x} \\
	\varepsilon_{yy} &= \frac{\partial v^*}{\partial y} \\
	\varepsilon_{xy} &= \frac{1}{2} \left( \frac{\partial u^*}{\partial y} + \frac{\partial v^*}{\partial x}  \right)
\end{align}

\clearpage

\section{Численное приближение}

Схема расчета:
\begin{itemize}
	\item Считаем деформации.
	\item Считаем напряжения.
	\item Добавляем внешние силы. Получаем $dv$.
	\item Делаем шаг по $dt$, пересчитываем скорости.
	\item Делаем шаг по $dt$, пересчитываем координаты.
\end{itemize}

Приближаем векторы $r_x, r_y$. В точке с индексами $(n, k)$:
\begin{equation}
	\frac{\partial w}{\partial x} \sim \frac{w_{n+1,k} - w_{n-1, k}}{2h} \\
	\frac{\partial w}{\partial y} \sim \frac{w_{n,k+1} - w_{n, k-1}}{2h}
\end{equation}

Введем векторы 
\begin{equation}
	\overline{\delta}_x = \begin{pmatrix}
				u_{n+1, k} - u_{n-1,k} \\
				v_{n+1, k} - v_{n-1,k} \\
				w_{n+1, k} - w_{n-1,k} 
				\end{pmatrix} \\
	\overline{\delta}_y = \begin{pmatrix}
				u_{n, k+1} - u_{n,k-1} \\
				v_{n, k+1} - v_{n,k-1} \\
				w_{n, k+1} - w_{n,k-1} 
				\end{pmatrix} \\
\end{equation}

Спроецируем их на наш локальный базис $\overline{r}_x, \overline{r}_y$:
\begin{align}
	\Pi_{xx} &= \frac{(\overline{\delta}_x, \overline{r}_x)}{\sqrt{(\overline{r}_x, \overline{r}_x)}} &\quad 
		\Pi_{yy} &= \frac{(\overline{\delta}_y, \overline{r}_y)}{\sqrt{(\overline{r}_y, \overline{r}_y)}} \\
	\Pi_{xy} &= \frac{(\overline{\delta}_x, \overline{r}_y)}{\sqrt{(\overline{r}_y, \overline{r}_y)}} &\quad 
		\Pi_{yx} &= \frac{(\overline{\delta}_y, \overline{r}_x)}{\sqrt{(\overline{r}_x, \overline{r}_x)}} 
\end{align}

Приближаем деформации:

\begin{equation}
	\varepsilon_{xx} \sim \frac{1}{2h}\Pi_{xx},\ \varepsilon_{yy} \sim \frac{1}{2h}\Pi_{yy}, 
		\varepsilon_{xy} \sim \frac{1}{4h}(\Pi_{xy} + \Pi_{yx})
\end{equation}

Зная деформации, получаем напряжения $\sigma_{ij}$, затем приращения скоростей. Проецируем их обратно в трёхмерное пространство:
\begin{align}
	\frac{\partial u}{\partial t} &= \frac{1}{\sqrt{1 + \left( \frac{\partial w}{\partial x} \right)^2 }} \frac{\partial v_x}{\partial t} \\
	\frac{\partial v}{\partial t} &= \frac{1}{\sqrt{1 + \left( \frac{\partial w}{\partial y} \right)^2 }} \frac{\partial v_y}{\partial t} \\
	\frac{\partial w}{\partial t} &= \frac{\partial w}{\partial x} \frac{\partial u}{\partial t} 
			+ \frac{\partial w}{\partial y} \frac{\partial v}{\partial t}
\end{align}
 























\end{document}
